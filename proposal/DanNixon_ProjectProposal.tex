\documentclass[a4paper,11pt]{article}

% Set for specific document
\def\DOCTITLE{
  The Application of Robust Analysis Methods on Sparse Data for Mass-Resolved
  Neutron Spectroscopy}
\def\DOCAUTHOR{Daniel Nixon (120263697)}
\def\DOCDATE{\today}

% Set document attributes
\title{\DOCTITLE}
\author{\DOCAUTHOR}
\date{\DOCDATE}

\usepackage{fullpage}
\usepackage{scrextend}
\usepackage{titlesec}
\usepackage{fancyhdr}
\usepackage{hyperref}
\usepackage[style=numeric-comp,natbib=true]{biblatex}
\usepackage[nonumberlist,nopostdot]{glossaries}
\usepackage{glossary-mcols}
\usepackage[section]{placeins}
\usepackage{subcaption}
\usepackage{minted}
\usepackage{enumitem}
\usepackage{booktabs}
\usepackage{tabularx}
\usepackage{amsmath}
\usepackage{pgfgantt}

\addbibresource{DanNixon_ProjectProposal.bib}

% Deeply nested lists
\setlistdepth{9}
\setlist[itemize,1]{label=$\bullet$}
\setlist[itemize,2]{label=$\bullet$}
\setlist[itemize,3]{label=$\bullet$}
\setlist[itemize,4]{label=$\bullet$}
\setlist[itemize,5]{label=$\bullet$}
\setlist[itemize,6]{label=$\bullet$}
\setlist[itemize,7]{label=$\bullet$}
\setlist[itemize,8]{label=$\bullet$}
\setlist[itemize,9]{label=$\bullet$}
\renewlist{itemize}{itemize}{9}

% Setup headers and footers
\pagestyle{fancy}
\lhead{}
\chead{}
\rhead{\DOCDATE}
\rfoot{\thepage}
\cfoot{}
\lfoot{\DOCAUTHOR}

% Set header and footer sizes
\renewcommand{\headrulewidth}{0.1pt}
\renewcommand{\footrulewidth}{0.1pt}
\setlength{\headheight}{14pt}
\setlength{\headsep}{14pt}

% New page for each section
% \newcommand{\sectionbreak}{\clearpage}

% Glossary
\setglossarystyle{altlist}
\makeglossaries

\newacronym{MANTID}{MANTID}
  {Manipulation and Analysis Toolkit for Instrument Data}
\newacronym{MANSE}{MANSE}
  {mass selective neutron spectroscopy}
\newacronym{NQD}{NQD}
  {nuclear quantum dynamics}

\newcommand{\RArrow}{$\Rightarrow$}

\def\ResearchCitationCol{\begin{tabularx}{0.15\textwidth}[t]{@{}X@{}}}
\def\ResearchSummaryCol{\begin{tabularx}{0.3\textwidth}[t]{@{}X@{}}}
\def\ResearchRelevanceCol{\begin{tabularx}{0.5\textwidth}[t]{@{}X@{}}}

\begin{document}

\maketitle

\begin{abstract}
  This document will give an overview of the problem my project will aim to
  solve, the research I have undertaken into the problem and the means by which
  I intend to solve it.
\end{abstract}

% \tableofcontents

\section{Motivation}
\label{sec:motivation}

VESUVIO is an indirect geometry neutron spectrometer at the ISIS facility,
Rutherford Appleton Laboratory, UK. This instrument is currently the only one of
its type in the world.

Thanks to recent developments allowing orders of magnitude improvements in
accuracy and reproducibility of measured spectra the science done using VESUVIO
has taken two distinct paths; \gls*{NQD} and \gls*{MANSE}.

VESUVIO also has a diffraction bank which can be used in conjunction with
spectroscopic techniques, specifically there is scope to use it with Bayesian
techniques to derive a model for analysis of spectroscopic data.

Currently the data analysis tools available to be used with spectroscopic data
from \gls*{MANSE} experiments is compromised of a limited set of fitting
routines implemented withing the \gls*{MANTID} framework \cite{mantid} based on
legacy analysis software.

Improvement to these routines would allow the scientific community to further
capitalise on the instrumental improvements to VESUVIO, this will benefit both
the scientists directly and those that depend on the research they carry out.

Currently the data analysis workflow requires significant knowledge of the
sample, this is used to set a system of constraints and ties to the fitted
parameters of the model that is fitted to the experimental spectrum. This
further complicates and increases the time taken for analysis.

This area of potential improvement forms the scope for this project.

On a personal level this project also carries on from work I undertook on my
year in industry, where around a third of my time was spent contributing to the
current data analysis and calibration workflows within \gls*{MANTID}.

\section{Project Overview}
\label{sec:overview}

The aim of this project is to improve the data analysis routines for
\gls*{MANSE} on the VESUVIO instrument.

This can be broken down into three main objectives:

\subsection{Development of new fitting functions}
\label{sec:objective1}

This objective deals with the implementation of new fitting functions that are
used to produce a set of fitted parameters in a given model such that they give
the best description of the measured spectrum from the instrument.

Through conversation with Dr Matthew Krzystyniak, one of the instrument
scientists on VESUVIO, it has been decided that the main focus of this objective
should be the implementation of the multivariate Gaussian function which is
described in the general case by Andreani et al \cite{Andreani2001}.

The success of this objective will be indicated by the new fitting function
being implemented within \gls*{MANTID}, including relevant unit tests and
documentation.

\subsection{Bayesian analysis routines}
\label{sec:objective2}

This objective will allow a model composed of multiple fitting functions to be
generated using Bayesian methods, this reduces the information that must be
known about the sample before the data analysis stage and allows the data
analysis to be more robust against spurious signals in the data (for example
those from the container/sample environment material).

This will most likely use a FABADA \cite{fabada} driven approach since this is
already implemented within \gls*{MANTID}, however there is another common
approach to Bayesian analysis as described by D.S. Sivia \cite{sivia_bayes}
which remains an option should issues with the FABADA approach arise.

The implementation and documentation of a Bayesian routine within \gls*{MANTID}
that derives reasonable spectroscopic models will be indicative of the success
of this objective.

\subsection{Case studies}
\label{sec:objective3}

This objective consists of several case studies into the effectiveness of the
routines implemented as part of objectives one and two, using well defined
samples (both experimental and simulated) to provide an accurate evaluation of
the implemented fitting and analysis.

Samples of interest may include:

\begin{itemize}
  \itemsep 0em
  \item Bayesian analysis of MANSE models in the presence of unwanted $\gamma$
        resonances
  \item Identification of components in VESUVIO spectra due to non-Gaussian
        distributions
  \item Kinetics of slow chemical or physical reactions
  \item Samples with unknown compositions, identification of trace
        elements/impurities (augmented by diffraction where appropriate)
\end{itemize}

These case studies will make up a large section of the evaluation of the project
in the final write up.

\section{Background Research}
\label{sec:background}

% \begin{longtable}{p{.15\textwidth} p{.3\textwidth} p{.5\textwidth}}
\begin{tabular}{lll}
  \toprule
  Reference & Summary & Relevance \\
  \midrule

  \ResearchCitationCol
    \gls*{MANTID} project\\
    \cite{mantid}
  \end{tabularx} &
  \ResearchSummaryCol
    Documentation and source code for the \gls*{MANTID} project.
  \end{tabularx} &
  \ResearchRelevanceCol
    Provides reference material for the implementations of fit functions and
    algorithms within \gls*{MANTID}.\\
    Specifically this has provided information about the current
    implementation of the existing fit functions for Compton scattering which
    the additional fit functions will integrate with.
  \end{tabularx} \\
  \midrule

  \ResearchCitationCol
    Single particle dynamics in fluid and solid hydrogen sulphide\\
    \cite{Andreani2001}
  \end{tabularx} &
  \ResearchSummaryCol
    Paper describing the multivariate Gaussian function and its use in fitting
    the signal from Hydrogen in VESUVIO experimental data.
  \end{tabularx} &
  \ResearchRelevanceCol
    Provides the formalisation of the multivariate Gaussian function that will
    be implemented for objective one, including demonstration of its use on
    experimental data which can then be used to evaluate the function
    implemented in objective one.
  \end{tabularx} \\
  \midrule

  \ResearchCitationCol
    FABADA Goes MANTID\\
    \cite{FabadaGoesMantid}
  \end{tabularx} &
  \ResearchSummaryCol
    Paper describing the implementation of the FABADA fitting algorithm within
    \gls*{MANTID} and an example of its use in analysis of quasielastic
    neutron scattering experiments.
  \end{tabularx} &
  \ResearchRelevanceCol
    Provides details of the implementation of FABADA that will form the basis
    of the Bayesian analysis routines for objective two.\\
    Contains a real world example of use of FABADA within \gls*{MANTID} which
    greatly improves understanding of its use.
  \end{tabularx} \\
  \midrule

  \ResearchCitationCol
    FABADA\\
    \cite{fabada}
  \end{tabularx} &
  \ResearchSummaryCol
    Paper describing the FABADA fitting algorithm.
  \end{tabularx} &
  \ResearchRelevanceCol
    Provides details of the theory of the FABADA fitting algorithm which would
    prove useful when writing the Bayesian routines for objective two.
  \end{tabularx} \\
  \midrule

  \ResearchCitationCol
    Data analysis: a Bayesian tutorial\\
    \cite{sivia_bayes}
  \end{tabularx} &
  \ResearchSummaryCol
    Book describing the foundations of Bayesian analysis, from which many of
    the Bayesian routines used in neutron scattering originate from.
  \end{tabularx} &
  \ResearchRelevanceCol
    This book provides a good introduction to the field of Bayesian analysis
    and is well respected in the field of neutron scattering data analysis.\\
    Should further analysis above what is possible with FABADA be required or
    in the case of any unforeseen issues with the FABADA implementation, an
    alternative is described in this book.
  \end{tabularx} \\
  % \midrule

  \bottomrule
% \end{longtable}
\end{tabular}

\section{Work Plan}
\label{sec:plan}

\ganttset{%
  calendar week text={%
    \startday%
  }%
}

\begin{figure}[h]
  \centering

  \begin{ganttchart}[time slot format=isodate,
                     hgrid=true,
                     vgrid={draw=none, draw=none, draw=none, draw=none,
                            draw=none, draw=none, dotted},
                     x unit=0.1cm,
                     y unit title=.6cm,
                     y unit chart=.5cm,
                     link mid=.25, link bulge=1.3,
                     link/.style={-to, blue}]
                    {2016-02-01}{2016-05-08}
    \gantttitlecalendar{month=name, week} \\

    \ganttgroup{Objective 1 (04/03)}{2016-02-01}{2016-03-04} \\
    \ganttbar{Fit Function (21/02)}{2016-02-01}{2016-02-21} \\
    \ganttbar{Testing (28/02)}{2016-02-08}{2016-02-28} \\
    \ganttbar{Documentation (21/02)}{2016-02-15}{2016-02-21} \\
    \ganttbar{Workflow Integration (04/03)}{2016-02-22}{2016-03-04} \\

    \ganttgroup{Objective 2 (11/04)}{2016-03-07}{2016-04-11} \\
    \ganttbar{Bayesian Analysis (03/04)}{2016-03-07}{2016-04-03} \\
    \ganttbar{Testing (10/04)}{2016-03-14}{2016-04-10} \\
    \ganttbar{Documentation (03/04)}{2016-03-21}{2016-04-03} \\
    \ganttbar{Workflow Integration (11/04)}{2016-04-04}{2016-04-11} \\

    \ganttgroup{Objective 3 (08/05)}{2016-02-01}{2016-05-08} \\
    \ganttbar{Fitting Case Studies (13/03)}{2016-03-04}{2016-03-13} \\
    \ganttbar{Bayesian Case Studies (24/04)}{2016-04-11}{2016-04-24} \\
    \ganttbar{Poster (25/04)}{2016-04-11}{2016-04-25} \\
    \ganttbar{Dissertation (02/05)}{2016-02-01}{2016-05-02} \\

    \ganttmilestone{Poster Hand In (28/04)}{2016-04-28} \\
    \ganttmilestone{Dissertation Hand In (08/05)}{2016-05-08}

    \ganttlink{elem2}{elem11}
    \ganttlink{elem1}{elem4}

    \ganttlink{elem7}{elem12}
    \ganttlink{elem6}{elem9}

    \ganttlink{elem13}{elem15}
    \ganttlink{elem14}{elem16}
  \end{ganttchart}

  \caption{Work plan Gantt chart (numbers at top denote week beginning, due
           dates are given in brackets)}
  \label{fig:gantt_chart}
\end{figure}

Figure \ref{fig:gantt_chart} shows the plan for the work required for the
project, this splits the time according to the three objectives, with the due
date for each objective being the date at the end of the thicker black bars.

In each of the implementation objectives there is a set amount of time for the
actual implementation as well as testing and documentation that will be required
to have the implementations accepted into the \gls*{MANTID} project.

Despite the fact that the implementation of the fitting function will be done
before the Bayesian analysis, there is no strict dependency for this to be the
case.

There is around two weeks of contingency time between the estimated completion
data of the Bayesian analysis case study and the hand in date for the
dissertation, this should provide sufficient time to allow other tasks to run
over by no more than one week, however the estimated times given to tasks have
been over estimated and due to the lack of dependencies the majority of tasks
can be performed in parallel.

\printbibliography

\end{document}
